\documentclass[a4paper]{article}
\usepackage{a4wide,amssymb,epsfig,latexsym,multicol,array,hhline,fancyhdr,systeme}
\usepackage{vntex}
\usepackage{amsmath}
\usepackage{lastpage}
\usepackage[lined,boxed,commentsnumbered]{algorithm2e}
\usepackage{enumerate}
\usepackage{color}
\usepackage{graphicx}                            % Standard graphics package
\usepackage{tabularx, caption}
\usepackage{multirow}
\usepackage{rotating}
\usepackage{graphics}
\usepackage[utf8]{vietnam}
\usepackage{geometry}
\usepackage{setspace}
\usepackage{tikz}
\usetikzlibrary{arrows,snakes,backgrounds}
\usepackage{hyperref}
\hypersetup{urlcolor=blue,linkcolor=black,citecolor=black,colorlinks=true}
\usepackage{physics}
%\usepackage{pstcol} 								% PSTricks with the standard color package

\newtheorem{theorem}{{\bf Theorem}}
\newtheorem{property}{{\bf Property}}
\newtheorem{proposition}{{\bf Proposition}}
\newtheorem{corollary}[proposition]{{\bf Corollary}}
\newtheorem{lemma}[proposition]{{\bf Lemma}}

\AtBeginDocument{\renewcommand*\contentsname{Contents}}
\AtBeginDocument{\renewcommand*\refname{References}}
%\usepackage{fancyhdr}
\setlength{\headheight}{40pt}
\pagestyle{fancy}
\fancyhead{} % clear all header fields
\fancyhead[L]{
    \begin{tabular}{rl}
        \begin{picture}(25,15)(0,0)
            \put(0,-8){\includegraphics[width=8mm, height=8mm]{hcmut.png}}
            %\put(0,-8){\epsfig{width=10mm,figure=hcmut.eps}}
        \end{picture} &
        %\includegraphics[width=8mm, height=8mm]{hcmut.png} & %
        \begin{tabular}{l}
            \textbf{\bf \ttfamily University of Technology, Ho Chi Minh City} \\
            \textbf{\bf \ttfamily Faculty of Computer Science and Engineering}
        \end{tabular}
    \end{tabular}
}
\fancyhead[R]{
    \begin{tabular}{l}
        \tiny \bf \\
        \tiny \bf
    \end{tabular}  }
\fancyfoot{} % clear all footer fields
\fancyfoot[L]{\scriptsize \ttfamily Assignment for Discrete Structures for Computing - Academic year 2019 - 2020}
\fancyfoot[R]{\scriptsize \ttfamily Page {\thepage}/\pageref{LastPage}}
\renewcommand{\headrulewidth}{0.3pt}
\renewcommand{\footrulewidth}{0.3pt}


%%%
\setcounter{secnumdepth}{4}
\setcounter{tocdepth}{3}
\makeatletter
\newcounter {subsubsubsection}[subsubsection]
\renewcommand\thesubsubsubsection{\thesubsubsection .\@alph\c@subsubsubsection}
\newcommand\subsubsubsection{\@startsection{subsubsubsection}{4}{\z@}%
{-3.25ex\@plus -1ex \@minus -.2ex}%
{1.5ex \@plus .2ex}%
{\normalfont\normalsize\bfseries}}
\newcommand*\l@subsubsubsection{\@dottedtocline{3}{10.0em}{4.1em}}
\newcommand*{\subsubsubsectionmark}[1]{}
\makeatother


\begin{document}

    \begin{titlepage}
        \begin{center}
            VIETNAM NATIONAL UNIVERSITY, HO CHI MINH CITY \\
            UNIVERSITY OF TECHNOLOGY \\
            FACULTY OF COMPUTER SCIENCE AND ENGINEERING
        \end{center}

        \vspace{1cm}

        \begin{figure}[h!]
            \begin{center}
                \includegraphics[width=3cm]{hcmut.png}
            \end{center}
        \end{figure}

        \vspace{1cm}


        \begin{center}
            \begin{tabular}{c}
                \multicolumn{1}{l}{\textbf{{\Large DISCRETE STRUCTURES FOR COMPUTING (CO1007)}}}\\
                ~~                                                                               \\
                \hline
                \\
                \multicolumn{1}{l}{\textbf{{\Large Assignment}}}                               \\
                \\
                \textbf{{\Huge Relation - Counting - Probability}}\\
                \\
                \textbf{{\Huge and Graph}}\\
                \\
                \hline
            \end{tabular}
        \end{center}

        \vspace{3cm}

        \begin{table}[h]
            \begin{tabular}{rrl}
                \hspace{5 cm} & Advisor:  & Fullname                                      \\
                & Students: & Fullname of Student 1 - Student 1 ID numbers. \\
                &           & Fullname of Student 2 - Student 2 ID numbers. \\
            \end{tabular}
        \end{table}

        \begin{center}
        {\footnotesize HO CHI MINH CITY, SEPTEMBER 2020}
        \end{center}
    \end{titlepage}


%\thispagestyle{empty}

    \newpage
    \tableofcontents
    \newpage


%%%%%%%%%%%%%%%%%%%%%%%%%%%%%%%%%


    \section{Member list \& Workload}

    \begin{center}
        \begin{tabular}{|c|c|c|l|c|}
            \hline
            \textbf{No.}       & \textbf{Fullname}              & \textbf{Student ID}       & \textbf{Problems}                 & \textbf{Percentage of work} \\
            \hline
%%%%%Student 1%%%%%%%%%%
            \multirow{3}{*}{1} & \multirow{3}{*}{Lưu Quốc Bình} & \multirow{3}{*}{2033009}  & - Exercise 1 & \multirow{3}{*}{30\%}\\
            &                                &                           & Bonus: 1, 2, 3.                   &                             \\
            &                                &                           & - Probability: 1, 2, 3.           &                             \\
            \hline
%%%%%Student 2%%%%%%%%%%%
            \multirow{3}{*}{2} & \multirow{3}{*}{Nguyễn Văn B}  & \multirow{3}{*}{19181717} & - Relation \& Counting: 4, 5, 6& \multirow{3}{*}{20\%}\\
            &                                &                           & Bonus: 4, 5, 6.                   &                             \\
            &                                &                           & - Graph: 1, 2, 3, Bonus: 1, 2, 3. &                             \\
            \hline
%%%%%Student 1%%%%%%%%%%
            \multirow{3}{*}{1} & \multirow{3}{*}{Nguyễn Văn A}  & \multirow{3}{*}{19181716} & - Relation \& Counting: 1, 2, 3& \multirow{3}{*}{30\%}\\
            &                                &                           & Bonus: 1, 2, 3.                   &                             \\
            &                                &                           & - Probability: 1, 2, 3.           &                             \\
            \hline
%%%%%Student 1%%%%%%%%%%
            \multirow{3}{*}{1} & \multirow{3}{*}{Nguyễn Văn A}  & \multirow{3}{*}{19181716} & - Relation \& Counting: 1, 2, 3& \multirow{3}{*}{30\%}\\
            &                                &                           & Bonus: 1, 2, 3.                   &                             \\
            &                                &                           & - Probability: 1, 2, 3.           &                             \\
            \hline
%%%%%Student 3%%%%%%%%%%%
        \end{tabular}
    \end{center}

%%%%%%%%%%%%%%%%%%%%%%%%%%%%%%%%%


    \section{Relation \& Counting}

    \subsection{Problem 1}
    Write on the report a very detailed introduction to the IVPs Sys. (3) and
    the formulae of its exact solutions for general a, b, c, and d and initial condition R0 and J0. Then
    complete Tab. 2 for all possible cases of eigenvalues of general 2 × 2 matrix A

    \subsubsection{Method of solving system of differential equations}
    Consider system:
    \begin{equation}
        \systeme{
            R' = aR + bJ,
            J' = cR + dJ,
            R(0) = R_0 \, J(0)=J_0
        }\label{eq:equation2}
    \end{equation}
    written in vector form
    \begin{equation*}
        \vec{x'} = A\vec{X}
    \end{equation*}
    will have solution form
    \begin{equation*}
        \vec{X} = \vec{\eta}e^\(\lambda t\)
    \end{equation*}
    where \begin{matrix}\lambda\end{matrix} and \begin{matrix}\vec{\eta}\end{matrix} are eigenvalues and eigenvectors of the matrix A
    , and
    \begin{equation*}
        \vec{x'}=\begin{pmatrix}
            a & b \\
            c & d \\
        \end{pmatrix};
        \vec{X}(0)=\begin{pmatrix}
             R_0 \\
             J_0 \\
        \end{pmatrix}
    \end{equation*}
    We are going to be looking for two solutions \begin{matrix}\vec{X_1}(t)\end{matrix} and \begin{matrix}\vec{X_2}(t)\end{matrix} where the determinant of the matrix
    \begin{equation*}
        X = \begin{pmatrix}
            \vec{X_1} &  \vec{X_2} \\
        \end{pmatrix}
    \end{equation*}
    So, the first thing that we need to do is find the eigenvalues for the matrix.
    \begin{equation*}
        det(A - \lambda I)
        =
        \begin{vmatrix}
            a-\lambda & b         \\
            c         & d-\lambda \\
        \end{vmatrix} = 0
    \end{equation*}
    simplifying to
    \begin{equation*}
        \lambda^2 - (a+d)\lambda + ad - bc = 0
    \end{equation*}
    we have
    \begin{equation*}
        \Delta = [-(a+d)]^2 - 4(ad - bc)
    \end{equation*}
    general solution in this case will be
    \begin{equation}
        \vec{X}(t) = {C_1}e^{\lambda_1 t}\vec{\eta}^{(1)} + {C_2}e^{\lambda_2 t}\vec{\eta}^{(2)} \label{eq:equation}
    \end{equation}

    \subsubsubsection{\Delta > 0,\ Real\ Eigenvalues,\ (4)\ have\ 2\ solution \lambda_1, \lambda_2}
    With \begin{matrix}\lambda_1\end{matrix}, we’ll need to solve,
    \begin{equation}
        \begin{pmatrix}
            a-\lambda_1 & b         \\
            c         & d-\lambda_1 \\
        \end{pmatrix}
        \begin{pmatrix}
            \eta_1\\
            \eta_2\\
        \end{pmatrix}
        =
        \begin{pmatrix}
            0 \\
            0 \\
        \end{pmatrix}
        \Rightarrow
        \begin{pmatrix}
            (a - \lambda_1)\eta_1 + b\eta_2\\
            (d - \lambda_1)\eta_2 + c\eta_1\\
        \end{pmatrix}0
        \Rightarrow
        \eta_1
            =\dfrac{-b}{a-\lambda_1}\eta_2\label{eq:equation3}
            =\dfrac{d - \lambda_1}{-c}\eta_2\label{eq:equation3'}
    \end{equation}
    The eigenvector in this case is,
    \begin{equation}
        \vec{\eta^{(1)}}=
        \begin{pmatrix}
            \dfrac{-b}{a-\lambda_1}\eta_2 \\
            \eta_2 \\
        \end{pmatrix}
        , \eta_2\ \in\ R\label{eq:equation9}
    \end{equation}
    and the same with \begin{matrix}\lambda_2\end{matrix}
    \begin{equation*}
        \vec{\eta^{(2)}}=
        \begin{pmatrix}
            \dfrac{-b}{a-\lambda_2}\eta_2' \\
            \eta_2' \\
        \end{pmatrix}
        , \eta_2'\ \in\ R
    \end{equation*}
    \begin{equation*}
        \Leftrightarrow
        \vec{X}(t) =
            {C_1}e^{\lambda_1 t}
                        \begin{pmatrix}
                            \dfrac{-b}{a-\lambda_1}\eta_2 \\
                            \eta_2 \\
                        \end{pmatrix}
            +
            {C_2}e^{\lambda_2 t}
                        \begin{pmatrix}
                            \dfrac{-b}{a-\lambda_2}\eta_2' \\
                            \eta_2' \\
                        \end{pmatrix}
    \end{equation*}
    Now, we need to find the constants.
    To do this we simply need to apply the initial conditions.
    \begin{equation*}
        \begin{pmatrix}
            R_0 \\
            J_0 \\
        \end{pmatrix}
        =
        \vec{X}(0)
        =
        {C_1}e^{\lambda_1 t}
            \begin{pmatrix}
                \dfrac{-b}{a-\lambda_1}\eta_2 \\
                \eta_2 \\
            \end{pmatrix}
        +
        {C_2}e^{\lambda_2 t}
            \begin{pmatrix}
                \dfrac{-b}{a-\lambda_2}\eta_2' \\
                \eta_2' \\
            \end{pmatrix}
    \end{equation*}
    All we need to do now is multiply the constants through and we then get two equations (one for each row) that we can solve for the constants. This gives C1, and C2.
    \begin{equation*}
        \left.
        \begin{array}{ll}
            \dfrac{-b}{a-\lambda_1}\eta_2 C_1 + \dfrac{-b}{a-\lambda_2}\eta_2' C_2 = R_0\\
            \eta_2 C1 + \eta_2' C2 = J_0
        \end{array}
        \right \} \Rightarrow C1, C2
    \end{equation*}


    \subsubsubsection{\Delta < 0,\ Complex\ Eigenvalues,\ \lambda_\(1,2\)= p \pm qi }

    Following (3) using second equation, with \begin{matrix}\lambda =  p + qi\end{matrix} (choose negative or positive, in this case I choose positive.), we have
    \begin{equation}
        \eta_1=\dfrac{-b}{a - (p + qi)}\eta_2=\dfrac{d - (p + qi)}{c}\eta_2\label{eq:equation4}
    \end{equation}
    So, the first eigenvector is,
    \begin{equation*}
        \vec{\eta^{(1)}}=
        \begin{pmatrix}
            \dfrac{d - (p + qi)}{-c}\eta_2 \\
            \eta_2 \\
        \end{pmatrix}
        , \eta_2\ \in\ R
    \end{equation*}
    The solution corresponding to this eigenvalue and eigenvector is
    \begin{equation*}
        \vec{X_1}(t)=
        e^{(p + qi)t}
        \begin{pmatrix}
            \dfrac{d - (p + qi)}{-c}\eta_2 \\
            \eta_2 \\
        \end{pmatrix}
        , \eta_2\ \in\ R
    \end{equation*}
    \begin{equation*}
        \vec{X_1}(t)=
        e^{pt}e^{qit}
        \begin{pmatrix}
            \dfrac{d - (p + qi)}{-c}\eta_2 \\
            \eta_2 \\
        \end{pmatrix}
    \end{equation*}
    Apply Euler’s formula (https://tutorial.math.lamar.edu/Classes/DE/ComplexRoots.aspx#EulerFormula),
    \begin{equation*}
        \vec{X_1}(t)=
        e^{pt}
        (cos(qt) + i sin(qt))
        \begin{pmatrix}
            \dfrac{d - (p + qi)}{-c}\eta_2 \\
            \eta_2 \\
        \end{pmatrix}
    \end{equation*}
    Withdraw \begin{matrix}\eta_2\end{matrix},
    \begin{equation*}
        \vec{X_1}(t)=
        e^{pt}\eta_2
        (cos(qt) + i sin(qt))
        \begin{pmatrix}
            \dfrac{d - (p + qi)}{-c} \\
            1 \\
        \end{pmatrix}
    \end{equation*}
    Transformation steps,
    \begin{equation*}
        \vec{X_1}(t)=
        e^{pt}\eta_2
        (cos(qt) + i sin(qt))
        \begin{pmatrix}
            \dfrac{d - p - qi}{-c} \\
            1 \\
        \end{pmatrix}
    \end{equation*}
    \begin{equation*}
        \vec{X_1}(t)=
        e^{pt}\eta_2
        (cos(qt) + i sin(qt))
        \begin{pmatrix}
            \dfrac{(d - p) - qi}{-c} \\
            1 \\
        \end{pmatrix}
    \end{equation*}
    \begin{equation*}
        \vec{X_1}(t)=
        e^{pt}\eta_2
        \begin{pmatrix}
            \dfrac{[cos(qt) + i sin(qt)][(d - p) - qi]}{-c} \\
            (cos(qt) + i sin(qt)) \\
        \end{pmatrix}
    \end{equation*}
    \begin{equation*}
        \vec{X_1}(t)=
        e^{pt}\eta_2
        \begin{pmatrix}
            \dfrac{cos(qt)(d - p) + i sin(qt)(d - p) - q i cos(qt) - q i i sin(qt)) ]}{-c} \\
            (cos(qt) + i sin(qt)) \\
        \end{pmatrix}
    \end{equation*}
    \begin{equation*}
        \vec{X_1}(t)=
        e^{pt}\eta_2
        \begin{pmatrix}
            \dfrac{(d - p)cos(qt) + i (d - p) sin(qt) - q i cos(qt) + q sin(qt)) ]}{-c} \\
            (cos(qt) + i sin(qt)) \\
        \end{pmatrix}
    \end{equation*}
    \begin{equation*}
        \vec{X_1}(t)=
        e^{pt}\eta_2
        \begin{pmatrix}
            \dfrac{[(d - p)cos(qt) + q sin(qt)] + [i (d - p) sin(qt) - q i cos(qt) )]}{-c} \\
            (cos(qt) + i sin(qt)) \\
        \end{pmatrix}
    \end{equation*}
    \begin{equation*}
        \vec{X_1}(t)=
        e^{pt}\eta_2
        \begin{pmatrix}
            \dfrac{[(d - p)cos(qt) + q sin(qt)] + i[(d - p) sin(qt) - q cos(qt) )]}{-c} \\
            (cos(qt) + i sin(qt)) \\
        \end{pmatrix}
    \end{equation*}
    \begin{equation*}
        \vec{X_1}(t)=
            e^{pt}\eta_2
            \begin{pmatrix}
                \dfrac{[(d - p)cos(qt) + q sin(qt)]}{-c} \\
                cos(qt) \\
            \end{pmatrix}
            +
            e^{pt}\eta_2 i
            \begin{pmatrix}
                \dfrac{[(d - p) sin(qt) - q cos(qt) )]}{-c} \\
                sin(qt) \\
            \end{pmatrix}
    \end{equation*}
    \begin{equation*}
        \vec{X_1}(t)=
        e^{pt}
        \begin{pmatrix}
            \dfrac{\eta_2[(d - p)cos(qt) + q sin(qt)]}{-c} \\
            \eta_2 cos(qt) \\
        \end{pmatrix}
        +
        i e^{pt}
        \begin{pmatrix}
            \dfrac{\eta_2[(d - p) sin(qt) - q cos(qt) )]}{-c} \\
            \eta_2 sin(qt) \\
        \end{pmatrix}
    \end{equation*}
    \begin{equation*}
        \vec{X_1}(t)=\vec{u}(t) + i\vec{v}(t)
    \end{equation*}
    The general solution to this system then,
    \newline
    \newline
    \fbox {
        \begin{equation}
            \vec{X_1}(t)=
                C_1 e^{pt}
                \begin{pmatrix}
                    \dfrac{\eta_2[(d - p)cos(qt) + q sin(qt)]}{-c} \\
                    \eta_2 cos(qt) \\
                \end{pmatrix}
                +

                C_2 e^{pt}
                \begin{pmatrix}
                    \dfrac{\eta_2[(d - p) sin(qt) - q cos(qt) )]}{-c} \\
                    \eta_2 sin(qt) \\
                \end{pmatrix}\label{eq:equation5}
        \end{equation}
    }
    \newline
    \newline
    Now apply the initial condition and find the constants.
    \begin{equation*}
        \fbox {
            \begin{pmatrix}
                R_0 \\
                J_0 \\
            \end{pmatrix}
            =
            \vec{X}(0)
            =
                {C_1}
                \begin{pmatrix}
                    \dfrac{\eta_2 (d - p)}{-c} \\
                    \eta_2 \\
                \end{pmatrix}
            +
                {C_2}
                \begin{pmatrix}
                    \dfrac{\eta_2 (-q)}{-c}\eta_2 \\
                    0 \\
                \end{pmatrix}
        }
    \end{equation*}
    \begin{equation*}
        \left.
        \begin{array}{ll}
            \dfrac{\eta_2 (d - p)}{-c} C_1 + \dfrac{\eta_2 (-q)}{-c}\eta_2 C_2 = R_0\\
            \eta_2 C1 = J_0
        \end{array}
        \right \} \Rightarrow C1, C2
    \end{equation*}

    \subsubsubsection{\Delta = 0,\ Repeated\ Eigenvalues}
    This is the final case that we need to take a look at.
    In this section we are going to look at solutions to the system,
    \begin{equation*}
        \vec{x'} = A\vec{X}
    \end{equation*}
    where the eigenvalues are repeated eigenvalues.
    Since we are going to be working with systems in which
    A is a 2×2 matrix we will make that assumption from the start.
    So, the system will have a double eigenvalue,
    \begin{matrix}\lambda\end{matrix}.

    This presents us with a problem.
    We want two linearly independent solutions so that we can form a general solution.
    However, with a double eigenvalue we will have only one.
    Following \href{https://tutorial.math.lamar.edu/Classes/DE/RepeatedEigenvalues.aspx}{this article}, I have find form to find solution fo this case.
    First solution is,
    \begin{equation}
        \vec{X_1} = \vec{\eta}e^{\lambda t} =
        \begin{pmatrix}
            \dfrac{-b}{a-\lambda_1}\eta_2 \\
            \eta_2 \\
        \end{pmatrix}
        =
        \begin{pmatrix}
            \dfrac{d - \lambda_1}{-c}\eta_2 \\
            \eta_2 \\
        \end{pmatrix}
        , (\forall \eta_2 \in R), because (3), (4)\label{eq:equation8}
    \end{equation}
    And second solution will be
    \begin{equation}
        \vec{X_2} = t e^{\lambda t} \vec{\eta} + e^{\lambda t}\vec{\rho}\label{eq:equation6}
    \end{equation}
    will be a solution to the system provided $\vec{\rho}$ is a solution to
    \begin{equation}
        (A - \lambda I) \vec{\rho} = \vec{\eta}\label{eq:equation7}
    \end{equation}
    Transform (9),
    \begin{equation*}
        (9)
        \Rightarrow
        \begin{pmatrix}
            a-\lambda_1 & b         \\
            c         & d-\lambda_1 \\
        \end{pmatrix}
        \begin{pmatrix}
            \rho_1 \\
            \rho_2 \\
        \end{pmatrix}
        =
        \begin{pmatrix}
            \eta_1 \\
            \eta_2 \\
        \end{pmatrix}
    \end{equation*}
    dựa vào (4), va thực hiện biến đổi tương tự, ta có
    \begin{equation*}
        \Leftrightarrow
        \begin{pmatrix}
            a-\lambda_1 & b         \\
            c         & d-\lambda_1 \\
        \end{pmatrix}
        \begin{pmatrix}
            \vec{\rho_1} \\
            \vec{\rho_2} \\
        \end{pmatrix}
        =
        \begin{pmatrix}
            \dfrac{-b}{a-\lambda_1}\eta_2 \\
            \eta_2 \\
        \end{pmatrix}
        =
        \begin{pmatrix}
            \dfrac{d - \lambda_1}{-c}\eta_2 \\
            \eta_2 \\
        \end{pmatrix}
    \end{equation*}
    \begin{equation*}
        \Leftrightarrow
        \begin{pmatrix}
        (a-\lambda_1) \rho_1 + b \rho_2         \\
        (d-\lambda_1) \rho_2 + c \rho_1 \\
        \end{pmatrix}
        =
        \begin{pmatrix}
            \dfrac{-b}{a-\lambda_1}\eta_2 \\
            \eta_2 \\
        \end{pmatrix}
        =
        \begin{pmatrix}
            \dfrac{d - \lambda_1}{-c}\eta_2 \\
            \eta_2 \\
        \end{pmatrix}
    \end{equation*}
    \begin{equation*}
        \Rightarrow
        \rho_1
            = (\dfrac{\dfrac{-b}{a - \lambda_1}\eta_2 - b \rho_2}{a - \lambda_1})
            = (\dfrac{\dfrac{d - \lambda_1}{-c}\eta_2 -b \rho_2}{a - \lambda_1})
            , and \rho_2 \in R
    \end{equation*}
    hoặc
    \begin{equation*}
        \rho_1
            = (\dfrac{\dfrac{-b}{a - \lambda_1}\eta_2 -(d - \lambda_1) \rho_2}{c})
            = (\dfrac{\dfrac{d - \lambda_1}{-c}\eta_2 -(d - \lambda_1) \rho_2}{c})
            , and \rho_2 \in R
    \end{equation*}

    Công thức nghiệm $\vec{X}$ của hệ sẽ là
    \begin{equation*}
        \vec{X} = C_1 \vec{X_1} + C_2 \vec{X_2}
    \end{equation*}
    \begin{equation*}
        \Leftrightarrow
        \vec{X} = C_1 \vec{\eta}e^{\lambda t} + C_2(t e^{\lambda t} \vec{\eta} + e^{\lambda t} \vec{\rho})
    \end{equation*}
    \begin{equation*}
        \Leftrightarrow
        \begin{pmatrix}
            R \\
            J \\
        \end{pmatrix}
        =
        C_1
        \begin{pmatrix}
            \eta_1 \\
            \eta_2 \\
        \end{pmatrix}
        e^{\lambda t}
        +
        C_2(
            t e^{\lambda t}
            \begin{pmatrix}
                \eta_1 \\
                \eta_2 \\
            \end{pmatrix}
            +
            e^{\lambda t}
            \begin{pmatrix}
                \rho_1 \\
                \rho_2 \\
            \end{pmatrix}
        )
    \end{equation*}
    \begin{equation*}
        \Leftrightarrow
        \vec{X} =
        C_1
        \begin{pmatrix}
                \eta_1 \\
                \eta_2 \\
        \end{pmatrix}
        e^{\lambda t}
        +
        C_2 e^{\lambda t}(
            t
            \begin{pmatrix}
                \eta_1 \\
                \eta_2 \\
            \end{pmatrix}
            +
            \begin{pmatrix}
                \rho_1 \\
                \rho_2 \\
            \end{pmatrix}
        )
    \end{equation*}
    \begin{equation*}
        \fbox {
            \Leftrightarrow
            \vec{X} =
            C_1
            \begin{pmatrix}
                    \eta_1 \\
                    \eta_2 \\
            \end{pmatrix}
            e^{\lambda t}
            +
            C_2 e^{\lambda t}
                \begin{pmatrix}
                    t \eta_1 +  \rho_1 \\
                    t \eta_2 +  \rho_2 \\
                \end{pmatrix}
        }
    \end{equation*}
    Now apply the initial condition and find the constants.
    \begin{equation*}
        \Leftrightarrow
        \begin{pmatrix}
            R_0 \\
            J_0 \\
        \end{pmatrix}
        =
        C_1
        \begin{pmatrix}
            \eta_1 \\
            \eta_2 \\
        \end{pmatrix}
        e^{\lambda t}
        +
        C_2
            \begin{pmatrix}
                t \eta_1 +  \rho_1 \\
                t \eta_2 +  \rho_2 \\
            \end{pmatrix}
    \end{equation*}
    \begin{equation*}
        \fbox {
            \Rightarrow
            \left.
            \begin{array}{ll}
                \eta_1 C_1 + \rho_1 C_2 = R_0\\
                \eta_2 C_1 + \rho_2 C_2 = J_0\\
            \end{array}
            \right \} \Rightarrow C1, C2
        }
    \end{equation*}


    \subsection{Problem 2}
    ...

    \subsection{Bonus exercises}
    ...

%%%%%%%%%%%%%%%%%%%%%%%%%%%%%%%%%


    \section{Probabilty}

    \subsection{Problem 1}
    ...

    \subsection{Problem 2}
    ...

    \subsection{Bonus exercises}
    ...

%%%%%%%%%%%%%%%%%%%%%%%%%%%%%%%%%


    \section{Graph}

    \subsection{Problem 1}
    ...

    \subsection{Problem 2}
    ...

    \subsection{Bonus exercises}
    ...

    \begin{thebibliography}{80}


        \bibitem{bib1}
        ...


        \bibitem{bib2}
        ...


    \end{thebibliography}
\end{document}

