\documentclass[a4paper]{article}
\usepackage[subpreambles=true]{standalone}
\usepackage{a4wide,amssymb,epsfig,latexsym,multicol,array,hhline,fancyhdr,systeme}
\usepackage{vntex}
\usepackage{lastpage}
\usepackage[lined,boxed,commentsnumbered]{algorithm2e}
\usepackage{enumerate}
\usepackage{graphicx}                            % Standard graphics package
\usepackage{tabularx, caption}
\usepackage{multirow}
\usepackage{rotating}
\usepackage{graphics}
\usepackage[utf8]{vietnam}
\usepackage{import}
\usepackage{empheq}
\usepackage{setspace}
%\usetikzlibrary{arrows,snakes,backgrounds}
\usepackage{hyperref}
\hypersetup{urlcolor=blue,linkcolor=black,citecolor=black,colorlinks=true}
\usepackage{physics}
\usepackage{color}
\usepackage{amsmath}
\usepackage{tcolorbox}
%\usepackage{pstcol} 								% PSTricks with the standard color package
\usepackage{chngcntr}

\counterwithin*{equation}{section}
\counterwithin*{equation}{subsection}
\newtheorem{theorem}{{\bf Theorem}}
\newtheorem{property}{{\bf Property}}
\newtheorem{proposition}{{\bf Proposition}}
\newtheorem{corollary}[proposition]{{\bf Corollary}}
\newtheorem{lemma}[proposition]{{\bf Lemma}}

\AtBeginDocument{\renewcommand*\contentsname{Contents}}
\AtBeginDocument{\renewcommand*\refname{References}}
%\usepackage{fancyhdr}
\setlength{\headheight}{40pt}
\pagestyle{fancy}
\fancyhead{} % clear all header fields
\fancyhead[L]{
    \begin{tabular}{rl}
        \begin{picture}(25,15)(0,0)
            \put(0,-8){\includegraphics[width=8mm, height=8mm]{hcmut.png}}
            %\put(0,-8){\epsfig{width=10mm,figure=hcmut.eps}}
        \end{picture} &
        %\includegraphics[width=8mm, height=8mm]{hcmut.png} & %
        \begin{tabular}{l}
            \textbf{\bf \ttfamily University of Technology, Ho Chi Minh City} \\
            \textbf{\bf \ttfamily Faculty of Computer Science and Engineering}
        \end{tabular}
    \end{tabular}
}
\fancyhead[R]{
    \begin{tabular}{l}
        \tiny \bf \\
        \tiny \bf
    \end{tabular}  }
\fancyfoot{} % clear all footer fields
\fancyfoot[L]{\scriptsize \ttfamily Assignment for Discrete Structures for Computing - Academic year 2019 - 2020}
\fancyfoot[R]{\scriptsize \ttfamily Page {\thepage}/\pageref{LastPage}}
\renewcommand{\headrulewidth}{0.3pt}
\renewcommand{\footrulewidth}{0.3pt}
%%%
\setcounter{secnumdepth}{4}
\setcounter{tocdepth}{3}
\makeatletter
\newcounter {subsubsubsection}[subsubsection]
\renewcommand\thesubsubsubsection{\thesubsubsection .\@alph\c@subsubsubsection}
\newcommand\subsubsubsection{\@startsection{subsubsubsection}{4}{\z@}%
{-3.25ex\@plus -1ex \@minus -.2ex}%
{1.5ex \@plus .2ex}%
{\normalfont\normalsize\bfseries}}
\newcommand*\l@subsubsubsection{\@dottedtocline{3}{10.0em}{4.1em}}
\newcommand*{\subsubsubsectionmark}[1]{}
\makeatother


\begin{document}
    \begin{titlepage}
        \begin{center}
            VIETNAM NATIONAL UNIVERSITY, HO CHI MINH CITY \\
            UNIVERSITY OF TECHNOLOGY \\
            FACULTY OF COMPUTER SCIENCE AND ENGINEERING
        \end{center}
        \vspace{1cm}
        \begin{figure}[h!]
            \begin{center}
                \includegraphics[width=3cm]{hcmut.png}
            \end{center}\label{fig:figure}
        \end{figure}
        \vspace{1cm}
        \begin{center}
            \begin{tabular}{c}
                \multicolumn{1}{l}{\textbf{{\Large MATHEMATICAL MODELING (CO2011)}}}\\
                ~~                                                                               \\
                \hline
                \\
                \multicolumn{1}{l}{\textbf{{\Large Assignment}}}                               \\
                \\
                \textbf{{\Huge “Dynamics of Love”}}\\
                \\
                \hline
            \end{tabular}
        \end{center}
        \vspace{3cm}
        \begin{table}[h]
            \begin{tabular}{rrl}
                \hspace{5 cm} & Advisor:  & -                                      \\
                & Students: & Nguyễn Thị Thanh Uyên - 2133187. \\
                &           & Nguyễn Trung Nghĩa - 2010448 \\
                &           & Lưu Quốc Bình - 2033009 \\
            \end{tabular}\label{tab:table}
        \end{table}
        \begin{center}
        {\footnotesize HO CHI MINH CITY, SEPTEMBER 2022}
        \end{center}
    \end{titlepage}
%\thispagestyle{empty}
    \newpage
    \tableofcontents
    \newpage
%%%%%%%%%%%%%%%%%%%%%%%%%%%%%%%%%
    \section{Member list \& Workload}

    \begin{center}
        \begin{tabular}{|c|c|c|l|c|}
            \hline
            \textbf{No.}       & \textbf{Fullname}                      & \textbf{Student ID}       & \textbf{Problems}                 & \textbf{Percentage of work} \\
            \hline
%%%%%Student 1%%%%%%%%%%
            \multirow{3}{*}{1} & \multirow{3}{*}{Lưu Quốc Bình} & \multirow{3}{*}{2033009}  & - Exercise 1 & \multirow{3}{*}{70\%}\\
            &                                &                           & &                             \\
            &                                &                           & &                             \\
            \hline
%%%%%Student 2%%%%%%%%%%%
            \multirow{3}{*}{2} & \multirow{3}{*}{Nguyễn Trung Nghĩa}  & \multirow{3}{*}{2010448} &- Exercise 2 & \multirow{3}{*}{100\%}\\
            &                                &                           & &                             \\
            &                                &                           & &                             \\
            \hline
%%%%%Student 1%%%%%%%%%%
            \multirow{3}{*}{3} & \multirow{3}{*}{Nguyễn Thị Thanh Uyên}       & \multirow{3}{*}{2133187} &- Exercise 3 & \multirow{3}{*}{100\%}\\
            &                                &                           & &                             \\
            &                                &                           & &                             \\
            \hline
        \end{tabular}
    \end{center}

%%%%%%%%%%%%%%%%%%%%%%%%%%%%%%%%%


    \section{Problem 1}\label{subsec:problem-1}
    \import{./}{p1.tex}

    \section{Problem 2}\label{subsec:problem-2}
%    \import{./}{p2.tex}
    ----

    \section{Problem 3}\label{subsec:problem-3}
    \import{./}{p3.tex}

    \section{Problem 4}\label{subsec:problem-4}
%    \import{./}{p4.tex}
    ----

    \subsection{Problem 5}
%    \import{./}{p5.tex}
    ----
\end{document}

