\documentclass[a4paper]{article}
\usepackage{a4wide,amssymb,epsfig,latexsym,multicol,array,hhline,fancyhdr,systeme}
\usepackage{vntex}
\usepackage{amsmath}
\usepackage{lastpage}
\usepackage[lined,boxed,commentsnumbered]{algorithm2e}
\usepackage{enumerate}
\usepackage{color}
\usepackage{graphicx}                            % Standard graphics package
\usepackage{tabularx, caption}
\usepackage{multirow}
\usepackage{rotating}
\usepackage{graphics}
\usepackage[utf8]{vietnam}
\usepackage{empheq}
\usepackage{geometry}
\usepackage{setspace}
\usepackage{tikz}
\usetikzlibrary{arrows,snakes,backgrounds}
\usepackage{hyperref}
\hypersetup{urlcolor=blue,linkcolor=black,citecolor=black,colorlinks=true}
\usepackage{physics}
%\usepackage{pstcol} 								% PSTricks with the standard color package

\newtheorem{theorem}{{\bf Theorem}}
\newtheorem{property}{{\bf Property}}
\newtheorem{proposition}{{\bf Proposition}}
\newtheorem{corollary}[proposition]{{\bf Corollary}}
\newtheorem{lemma}[proposition]{{\bf Lemma}}

\AtBeginDocument{\renewcommand*\contentsname{Contents}}
\AtBeginDocument{\renewcommand*\refname{References}}
%\usepackage{fancyhdr}
\setlength{\headheight}{40pt}
\pagestyle{fancy}
\fancyhead{} % clear all header fields
\fancyhead[L]{
    \begin{tabular}{rl}
        \begin{picture}(25,15)(0,0)
            \put(0,-8){\includegraphics[width=8mm, height=8mm]{hcmut.png}}
            %\put(0,-8){\epsfig{width=10mm,figure=hcmut.eps}}
        \end{picture} &
        %\includegraphics[width=8mm, height=8mm]{hcmut.png} & %
        \begin{tabular}{l}
            \textbf{\bf \ttfamily University of Technology, Ho Chi Minh City} \\
            \textbf{\bf \ttfamily Faculty of Computer Science and Engineering}
        \end{tabular}
    \end{tabular}
}
\fancyhead[R]{
    \begin{tabular}{l}
        \tiny \bf \\
        \tiny \bf
    \end{tabular}  }
\fancyfoot{} % clear all footer fields
\fancyfoot[L]{\scriptsize \ttfamily Assignment for Discrete Structures for Computing - Academic year 2019 - 2020}
\fancyfoot[R]{\scriptsize \ttfamily Page {\thepage}/\pageref{LastPage}}
\renewcommand{\headrulewidth}{0.3pt}
\renewcommand{\footrulewidth}{0.3pt}


%%%
\setcounter{secnumdepth}{4}
\setcounter{tocdepth}{3}
\makeatletter
\newcounter {subsubsubsection}[subsubsection]
\renewcommand\thesubsubsubsection{\thesubsubsection .\@alph\c@subsubsubsection}
\newcommand\subsubsubsection{\@startsection{subsubsubsection}{4}{\z@}%
{-3.25ex\@plus -1ex \@minus -.2ex}%
{1.5ex \@plus .2ex}%
{\normalfont\normalsize\bfseries}}
\newcommand*\l@subsubsubsection{\@dottedtocline{3}{10.0em}{4.1em}}
\newcommand*{\subsubsubsectionmark}[1]{}
\makeatother


\begin{document}

    \begin{titlepage}
        \begin{center}
            VIETNAM NATIONAL UNIVERSITY, HO CHI MINH CITY \\
            UNIVERSITY OF TECHNOLOGY \\
            FACULTY OF COMPUTER SCIENCE AND ENGINEERING
        \end{center}

        \vspace{1cm}

        \begin{figure}[h!]
            \begin{center}
                \includegraphics[width=3cm]{hcmut.png}
            \end{center}
        \end{figure}

        \vspace{1cm}


        \begin{center}
            \begin{tabular}{c}
                \multicolumn{1}{l}{\textbf{{\Large DISCRETE STRUCTURES FOR COMPUTING (CO1007)}}}\\
                ~~                                                                               \\
                \hline
                \\
                \multicolumn{1}{l}{\textbf{{\Large Assignment}}}                               \\
                \\
                \textbf{{\Huge Relation - Counting - Probability}}\\
                \\
                \textbf{{\Huge and Graph}}\\
                \\
                \hline
            \end{tabular}
        \end{center}

        \vspace{3cm}

        \begin{table}[h]
            \begin{tabular}{rrl}
                \hspace{5 cm} & Advisor:  & Fullname                                      \\
                & Students: & Fullname of Student 1 - Student 1 ID numbers. \\
                &           & Fullname of Student 2 - Student 2 ID numbers. \\
            \end{tabular}
        \end{table}

        \begin{center}
        {\footnotesize HO CHI MINH CITY, SEPTEMBER 2020}
        \end{center}
    \end{titlepage}


%\thispagestyle{empty}

    \newpage
    \tableofcontents
    \newpage


%%%%%%%%%%%%%%%%%%%%%%%%%%%%%%%%%


    \section{Member list \& Workload}

    \begin{center}
        \begin{tabular}{|c|c|c|l|c|}
            \hline
            \textbf{No.}       & \textbf{Fullname}              & \textbf{Student ID}       & \textbf{Problems}                 & \textbf{Percentage of work} \\
            \hline
%%%%%Student 1%%%%%%%%%%
            \multirow{3}{*}{1} & \multirow{3}{*}{Lưu Quốc Bình} & \multirow{3}{*}{2033009}  & - Exercise 1 & \multirow{3}{*}{30\%}\\
            &                                &                           & Bonus: 1, 2, 3.                   &                             \\
            &                                &                           & - Probability: 1, 2, 3.           &                             \\
            \hline
%%%%%Student 2%%%%%%%%%%%
            \multirow{3}{*}{2} & \multirow{3}{*}{Nguyễn Văn B}  & \multirow{3}{*}{19181717} & - Relation \& Counting: 4, 5, 6& \multirow{3}{*}{20\%}\\
            &                                &                           & Bonus: 4, 5, 6.                   &                             \\
            &                                &                           & - Graph: 1, 2, 3, Bonus: 1, 2, 3. &                             \\
            \hline
%%%%%Student 1%%%%%%%%%%
            \multirow{3}{*}{1} & \multirow{3}{*}{Nguyễn Văn A}  & \multirow{3}{*}{19181716} & - Relation \& Counting: 1, 2, 3& \multirow{3}{*}{30\%}\\
            &                                &                           & Bonus: 1, 2, 3.                   &                             \\
            &                                &                           & - Probability: 1, 2, 3.           &                             \\
            \hline
%%%%%Student 1%%%%%%%%%%
            \multirow{3}{*}{1} & \multirow{3}{*}{Nguyễn Văn A}  & \multirow{3}{*}{19181716} & - Relation \& Counting: 1, 2, 3& \multirow{3}{*}{30\%}\\
            &                                &                           & Bonus: 1, 2, 3.                   &                             \\
            &                                &                           & - Probability: 1, 2, 3.           &                             \\
            \hline
%%%%%Student 3%%%%%%%%%%%
        \end{tabular}
    \end{center}

%%%%%%%%%%%%%%%%%%%%%%%%%%%%%%%%%


    \section{Relation \& Counting}

    \subsection{Problem 1}
    Write on the report a very detailed introduction to the IVPs Sys. (3) and
    the formulae of its exact solutions for general a, b, c, and d and initial condition R0 and J0. Then
    complete Tab. 2 for all possible cases of eigenvalues of general 2 × 2 matrix A

    \subsubsection{Method of solving system of differential equations}
    Xét hệ:
    \begin{equation}
        \systeme{
            R' = aR + bJ,
            J' = cR + dJ,
            R(0) = R_0 \, J(0)=J_0
        }\label{eq:equation2}
    \end{equation}
    dược viết dưới dạng vector
    \begin{equation*}
        \vec{x'} = A\vec{X}
    \end{equation*}
    sẽ có công thức nghiệm là
    \begin{equation*}
        \vec{X} = \vec{\eta}e^\(\lambda t\)
    \end{equation*}
    ở lây \begin{matrix}\lambda\end{matrix} and \begin{matrix}\vec{\eta}\end{matrix} là eigenvalues và eigenvectors của , ma trận A
    , và
    \begin{equation*}
        \vec{x'}=\begin{pmatrix}
            a & b \\
            c & d \\
        \end{pmatrix};
        \vec{X}(t)=\begin{pmatrix}
             R(t) \\
             J(t) \\
        \end{pmatrix};
        \vec{X}(0)=\begin{pmatrix}
             R_0 \\
             J_0 \\
        \end{pmatrix}
    \end{equation*}
    Đầu tiên, chúng ta tìm eigenvalues của ma trận A.
    \begin{equation*}
        det(A - \lambda I)
        =
        \begin{vmatrix}
            a-\lambda & b         \\
            c         & d-\lambda \\
        \end{vmatrix} = 0
    \end{equation*}
    triển khai và rút gọn,
    \begin{equation*}
        \lambda^2 - (a+d)\lambda + ad - bc = 0
    \end{equation*}
    tính delta,
    \begin{equation*}
        \Delta = [-(a+d)]^2 - 4(ad - bc)
    \end{equation*}
    dạng chung của nghiệm sẽ là
    \begin{equation}
        \vec{X}(t)
            = C_1 X_1 + C_2 X_2 \label{eq:equation}
    \end{equation}

    \subsubsubsection{\Delta > 0,\ (Real\ Eigenvalues),\ (4)\ có\ 2\ nghiệm \lambda_1, \lambda_2}
    Với \begin{matrix}\lambda_1\end{matrix}, chúng ta sẽ xử lý như sau,
    \begin{equation}
        \begin{pmatrix}
            a-\lambda_1 & b         \\
            c         & d-\lambda_1 \\
        \end{pmatrix}
        \begin{pmatrix}
            \eta_1\\
            \eta_2\\
        \end{pmatrix}
        =
        \begin{pmatrix}
            0 \\
            0 \\
        \end{pmatrix}
        \Rightarrow
        \begin{pmatrix}
            (a - \lambda_1)\eta_1 + b\eta_2\\
            (d - \lambda_1)\eta_2 + c\eta_1\\
        \end{pmatrix}0
        \Rightarrow
        \eta_1
            =\dfrac{-b}{a-\lambda_1}\eta_2\label{eq:equation3}
            =\dfrac{d - \lambda_1}{-c}\eta_2\label{eq:equation3'}
    \end{equation}
    eigenvector trong trường hợp này sẽ là,
    \begin{equation}
        \vec{\eta^{(1)}}=
        \begin{pmatrix}
            \dfrac{-b}{a-\lambda_1}\eta_2 \\
            \eta_2 \\
        \end{pmatrix}
        , \eta_2\ \in\ R\label{eq:equation9}
    \end{equation}
    tương tự với \begin{matrix}\lambda_2\end{matrix}
    \begin{equation*}
        \vec{\eta^{(2)}}=
        \begin{pmatrix}
            \dfrac{-b}{a-\lambda_2}\eta_2' \\
            \eta_2' \\
        \end{pmatrix}
        , \eta_2'\ \in\ R
    \end{equation*}
    \begin{center}
        \fbox {
            \begin{equation*}
                \Leftrightarrow
                \vec{X}(t) =
                    {C_1}e^{\lambda_1 t}
                                \begin{pmatrix}
                                    \dfrac{-b}{a-\lambda_1}\eta_2 \\
                                    \eta_2 \\
                                \end{pmatrix}
                    +
                    {C_2}e^{\lambda_2 t}
                                \begin{pmatrix}
                                    \dfrac{-b}{a-\lambda_2}\eta_2' \\
                                    \eta_2' \\
                                \end{pmatrix}
            \end{equation*}
        }
    \end{center}
    Tìm các hằng số $C_1, C_2$ với điều kiện khởi tạo $R_0, J_0$
    \begin{equation*}
        \begin{pmatrix}
            R_0 \\
            J_0 \\
        \end{pmatrix}
        =
        \vec{X}(0)
        =
        {C_1}e^{\lambda_1 t}
            \begin{pmatrix}
                \dfrac{-b}{a-\lambda_1}\eta_2 \\
                \eta_2 \\
            \end{pmatrix}
        +
        {C_2}e^{\lambda_2 t}
            \begin{pmatrix}
                \dfrac{-b}{a-\lambda_2}\eta_2' \\
                \eta_2' \\
            \end{pmatrix}
    \end{equation*}
    Nhân các hệ số với nhau, ta sẽ có hệ phương trình với biến cần tìm là $C_1, C_2$
    \begin{center}
        \fbox {
            \begin{equation*}
                \left.
                \begin{array}{ll}
                    \dfrac{-b}{a-\lambda_1}\eta_2 C_1 + \dfrac{-b}{a-\lambda_2}\eta_2' C_2 = R_0\\
                    \eta_2 C1 + \eta_2' C2 = J_0
                \end{array}
                \right \} \Rightarrow C1, C2
            \end{equation*}
        }
        , với $\eta_2 \in R$ tuỳ ý
        , $\lambda_1, \lambda_2$ là eigenvalues
    \end{center}


    \subsubsubsection{\Delta < 0,\ Complex\ Eigenvalues,\ \lambda_\(1,2\)= p \pm qi }

    Dựa vào (3), với \begin{matrix}\lambda =  p + qi\end{matrix} (chọn 1 nghiệm phúc với dấu âm hoặc dương , trong trường hợp này em/tôi chọn nghiệm phức dương.),
    chúng ta có
    \begin{equation}
        \eta_1=\dfrac{-b}{a - (p + qi)}\eta_2=\dfrac{d - (p + qi)}{c}\eta_2\label{eq:equation4}
    \end{equation}
    eigenvector đầu tiên là,
    \begin{equation*}
        \vec{\eta^{(1)}}=
        \begin{pmatrix}
            \dfrac{d - (p + qi)}{-c}\eta_2 \\
            \eta_2 \\
        \end{pmatrix}
        , \eta_2\ \in\ R
    \end{equation*}
    Dựa vào công thức nghiệm ở phía trên, $\vec{X_1}$ là
    \begin{equation*}
        \vec{X_1}(t)=
        e^{(p + qi)t}
        \begin{pmatrix}
            \dfrac{d - (p + qi)}{-c}\eta_2 \\
            \eta_2 \\
        \end{pmatrix}
        , \eta_2\ \in\ R
    \end{equation*}
    \begin{equation*}
        \vec{X_1}(t)=
        e^{pt}e^{qit}
        \begin{pmatrix}
            \dfrac{d - (p + qi)}{-c}\eta_2 \\
            \eta_2 \\
        \end{pmatrix}
    \end{equation*}
    Áp dụng công thức Euler (https://tutorial.math.lamar.edu/Classes/DE/ComplexRoots.aspx#EulerFormula),
    \begin{equation*}
        \vec{X_1}(t)=
        e^{pt}
        (cos(qt) + i sin(qt))
        \begin{pmatrix}
            \dfrac{d - (p + qi)}{-c}\eta_2 \\
            \eta_2 \\
        \end{pmatrix}
    \end{equation*}
    Rút gọn \begin{matrix}\eta_2\end{matrix},
    \begin{equation*}
        \vec{X_1}(t)=
        e^{pt}\eta_2
        (cos(qt) + i sin(qt))
        \begin{pmatrix}
            \dfrac{d - (p + qi)}{-c} \\
            1 \\
        \end{pmatrix}
    \end{equation*}
    Các bước biến đổi,
    \begin{equation*}
        \vec{X_1}(t)=
        e^{pt}\eta_2
        (cos(qt) + i sin(qt))
        \begin{pmatrix}
            \dfrac{d - p - qi}{-c} \\
            1 \\
        \end{pmatrix}
    \end{equation*}
    \begin{equation*}
        \vec{X_1}(t)=
        e^{pt}\eta_2
        (cos(qt) + i sin(qt))
        \begin{pmatrix}
            \dfrac{(d - p) - qi}{-c} \\
            1 \\
        \end{pmatrix}
    \end{equation*}
    \begin{equation*}
        \vec{X_1}(t)=
        e^{pt}\eta_2
        \begin{pmatrix}
            \dfrac{[cos(qt) + i sin(qt)][(d - p) - qi]}{-c} \\
            (cos(qt) + i sin(qt)) \\
        \end{pmatrix}
    \end{equation*}
    \begin{equation*}
        \vec{X_1}(t)=
        e^{pt}\eta_2
        \begin{pmatrix}
            \dfrac{cos(qt)(d - p) + i sin(qt)(d - p) - q i cos(qt) - q i i sin(qt)) ]}{-c} \\
            (cos(qt) + i sin(qt)) \\
        \end{pmatrix}
    \end{equation*}
    \begin{equation*}
        \vec{X_1}(t)=
        e^{pt}\eta_2
        \begin{pmatrix}
            \dfrac{(d - p)cos(qt) + i (d - p) sin(qt) - q i cos(qt) + q sin(qt)) ]}{-c} \\
            (cos(qt) + i sin(qt)) \\
        \end{pmatrix}
    \end{equation*}
    \begin{equation*}
        \vec{X_1}(t)=
        e^{pt}\eta_2
        \begin{pmatrix}
            \dfrac{[(d - p)cos(qt) + q sin(qt)] + [i (d - p) sin(qt) - q i cos(qt) )]}{-c} \\
            (cos(qt) + i sin(qt)) \\
        \end{pmatrix}
    \end{equation*}
    \begin{equation*}
        \vec{X_1}(t)=
        e^{pt}\eta_2
        \begin{pmatrix}
            \dfrac{[(d - p)cos(qt) + q sin(qt)] + i[(d - p) sin(qt) - q cos(qt) )]}{-c} \\
            (cos(qt) + i sin(qt)) \\
        \end{pmatrix}
    \end{equation*}
    \begin{equation*}
        \vec{X_1}(t)=
            e^{pt}\eta_2
            \begin{pmatrix}
                \dfrac{[(d - p)cos(qt) + q sin(qt)]}{-c} \\
                cos(qt) \\
            \end{pmatrix}
            +
            e^{pt}\eta_2 i
            \begin{pmatrix}
                \dfrac{[(d - p) sin(qt) - q cos(qt) )]}{-c} \\
                sin(qt) \\
            \end{pmatrix}
    \end{equation*}
    \begin{equation*}
        \vec{X_1}(t)=
        e^{pt}
        \begin{pmatrix}
            \dfrac{\eta_2[(d - p)cos(qt) + q sin(qt)]}{-c} \\
            \eta_2 cos(qt) \\
        \end{pmatrix}
        +
        i e^{pt}
        \begin{pmatrix}
            \dfrac{\eta_2[(d - p) sin(qt) - q cos(qt) )]}{-c} \\
            \eta_2 sin(qt) \\
        \end{pmatrix}
    \end{equation*}
    \begin{equation*}
        \vec{X_1}(t)=\vec{u}(t) + i\vec{v}(t)
    \end{equation*}
    Vậy công thức nghiệm chung trong trường hợp $\Delta < 0$ sẽ là
    \newline
    \newline
    \framebox {
        \begin{equation}
            \vec{X_1}(t)=
                C_1 e^{pt}
                \begin{pmatrix}
                    \dfrac{\eta_2[(d - p)cos(qt) + q sin(qt)]}{-c} \\
                    \eta_2 cos(qt) \\
                \end{pmatrix}
                +

                C_2 e^{pt}
                \begin{pmatrix}
                    \dfrac{\eta_2[(d - p) sin(qt) - q cos(qt) )]}{-c} \\
                    \eta_2 sin(qt) \\
                \end{pmatrix}\label{eq:equation5}
        \end{equation}
    }
    \newline
    \newline
    Áp dụng điều kiện khỡi tạo để tìm các hằng số $C_1, C_2$.
    \begin{equation*}
        \begin{pmatrix}
            R_0 \\
            J_0 \\
        \end{pmatrix}
        =
        \vec{X}(0)
        =
            {C_1}
            \begin{pmatrix}
                \dfrac{\eta_2 (d - p)}{-c} \\
                \eta_2 \\
            \end{pmatrix}
        +
            {C_2}
            \begin{pmatrix}
                \dfrac{\eta_2 (-q)}{-c}\eta_2 \\
                0 \\
            \end{pmatrix}
    \end{equation*}
    \begin{center}
        \fbox {
            \begin{equation*}
                \left.
                \begin{array}{ll}
                    \dfrac{\eta_2 (d - p)}{-c} C_1 + \dfrac{\eta_2 (-q)}{-c}\eta_2 C_2 = R_0\\
                    \eta_2 C1 = J_0
                \end{array}
                \right \} \Rightarrow C1, C2
            \end{equation*}
        }
    \end{center}
    \begin{center}
        Với
        $\eta_1, \eta_2$ tính từ (5),
        p là $Re(\lambda_1)$,
        q là $Im(\lambda_1)$.
    \end{center}


    \subsubsubsection{\Delta = 0,\ Repeated\ Eigenvalues}
    Mô tả vấn đề.
    Chúng ta mong muốn có 2 nghiệm phân biệt, không phụ thuộc nhau để tạo thành một nghiệm chung
    Tuy nhiên, trong trường hợp này, eigenvalues là nghiệm kép.
    Thep \href{https://tutorial.math.lamar.edu/Classes/DE/RepeatedEigenvalues.aspx}{bài viết này}, em tìm được cách giải quyết như sau.
    \newline
    \newline
    Nghiệm đầu tiên, chúng ta sẽ làm tương tự như trường hợp $\Delta > 0$,
    \begin{equation}
        \vec{X_1} = \vec{\eta} e^{\lambda t} =
        \begin{pmatrix}
            \eta_1 \\
            \eta_2 \\
        \end{pmatrix}
        e^{\lambda t}
        =
        \begin{pmatrix}
            \dfrac{-b}{a-\lambda_1}\eta_2 \\
            \eta_2 \\
        \end{pmatrix}
        e^{\lambda t}
        =
        \begin{pmatrix}
            \dfrac{d - \lambda_1}{-c}\eta_2 \\
            \eta_2 \\
        \end{pmatrix}
        e^{\lambda t}
        , (\forall \eta_2 \in R), because (3), (4)\label{eq:equation8}
    \end{equation}
    sau khi tính được $\eta$, ta sẽ sử dụng $\eta$ để tính nghiệm 2.
    \newline
    \newline
    Nghiệm thứ hai,
    \begin{equation}
        \vec{X_2} = t e^{\lambda t} \vec{\eta} + e^{\lambda t}\vec{\rho}\label{eq:equation6}
    \end{equation}
    \newline
    \newline
    \newline
    với $\vec{\rho}$ sẽ thoả
    \begin{equation*}
        (A - \lambda I) \vec{\rho} = \vec{\eta}
    \end{equation*}
    \begin{equation*}
        \Leftrightarrow
        \begin{pmatrix}
            a-\lambda_1 & b         \\
            c         & d-\lambda_1 \\
        \end{pmatrix}
        \begin{pmatrix}
            \rho_1 \\
            \rho_2 \\
        \end{pmatrix}
        =
        \begin{pmatrix}
            \eta_1 \\
            \eta_2 \\
        \end{pmatrix}
    \end{equation*}
    dựa vào (4), va thực hiện biến đổi tương tự, ta có
    \begin{equation*}
        \Leftrightarrow
        \begin{pmatrix}
            a-\lambda_1 & b         \\
            c         & d-\lambda_1 \\
        \end{pmatrix}
        \begin{pmatrix}
            \vec{\rho_1} \\
            \vec{\rho_2} \\
        \end{pmatrix}
        =
        \begin{pmatrix}
            \dfrac{-b}{a-\lambda_1}\eta_2 \\
            \eta_2 \\
        \end{pmatrix}
        =
        \begin{pmatrix}
            \dfrac{d - \lambda_1}{-c}\eta_2 \\
            \eta_2 \\
        \end{pmatrix}
    \end{equation*}
    \begin{equation*}
        \Leftrightarrow
        \begin{pmatrix}
        (a-\lambda_1) \rho_1 + b \rho_2         \\
        (d-\lambda_1) \rho_2 + c \rho_1 \\
        \end{pmatrix}
        =
        \begin{pmatrix}
            \dfrac{-b}{a-\lambda_1}\eta_2 \\
            \eta_2 \\
        \end{pmatrix}
        =
        \begin{pmatrix}
            \dfrac{d - \lambda_1}{-c}\eta_2 \\
            \eta_2 \\
        \end{pmatrix}
    \end{equation*}
    Suy ra cong thức $\rho_1$ dựa vào $\rho_2 \in R$ tuỳ ý
    \begin{equation}
        \Rightarrow
        \rho_1
            = (\dfrac{\dfrac{-b}{a - \lambda_1}\eta_2 - b \rho_2}{a - \lambda_1})
            , and \rho_2 \in R\label{eq:equation7}
    \end{equation}
    hoặc
    \begin{equation}
        \rho_1
            = (\dfrac{\dfrac{d - \lambda_1}{-c}\eta_2 -b \rho_2}{a - \lambda_1})
            , and \rho_2 \in R\label{eq:equation10}
    \end{equation}
    hoặc
    \begin{equation}
        \rho_1
            = (\dfrac{\dfrac{-b}{a - \lambda_1}\eta_2 -(d - \lambda_1) \rho_2}{c})
            , and \rho_2 \in R\label{eq:equation11}
    \end{equation}
    hoặc
    \begin{equation}
        \rho_1
            = (\dfrac{\dfrac{d - \lambda_1}{-c}\eta_2 -(d - \lambda_1) \rho_2}{c})
            , and \rho_2 \in R\label{eq:equation12}
    \end{equation}

    Công thức nghiệm $\vec{X}$ của hệ sẽ là
    \begin{equation*}
        \vec{X} = C_1 \vec{X_1} + C_2 \vec{X_2}
    \end{equation*}
    \begin{equation*}
        \Leftrightarrow
        \vec{X} = C_1 \vec{\eta}e^{\lambda t} + C_2(t e^{\lambda t} \vec{\eta} + e^{\lambda t} \vec{\rho})
    \end{equation*}
    \begin{equation*}
        \Leftrightarrow
        \vec{X} =
        C_1
        \begin{pmatrix}
            \eta_1 \\
            \eta_2 \\
        \end{pmatrix}
        e^{\lambda t}
        +
        C_2 e^{\lambda t}(
            t
            \begin{pmatrix}
                \eta_1 \\
                \eta_2 \\
            \end{pmatrix}
            +
            \begin{pmatrix}
                \rho_1 \\
                \rho_2 \\
            \end{pmatrix}
        )
    \end{equation*}
    \begin{equation*}
        \fbox {
            \Leftrightarrow
            \vec{X} =
            C_1
            \begin{pmatrix}
                    \eta_1 \\
                    \eta_2 \\
            \end{pmatrix}
            e^{\lambda t}
            +
            C_2 e^{\lambda t}
                \begin{pmatrix}
                    t \eta_1 +  \rho_1 \\
                    t \eta_2 +  \rho_2 \\
                \end{pmatrix}
        }
    \end{equation*}
    Áp dụng điều kiện khỡi tạo để tìm các hằng số $C_1, C_2$.
    \begin{equation*}
        \Leftrightarrow
        \begin{pmatrix}
            R_0 \\
            J_0 \\
        \end{pmatrix}
        =
        C_1
        \begin{pmatrix}
            \eta_1 \\
            \eta_2 \\
        \end{pmatrix}
        e^{\lambda t}
        +
        C_2
            \begin{pmatrix}
                t \eta_1 +  \rho_1 \\
                t \eta_2 +  \rho_2 \\
            \end{pmatrix}
    \end{equation*}
    \begin{equation*}
        \fbox {
            \Rightarrow
            \left.
            \begin{array}{ll}
                \eta_1 C_1 + \rho_1 C_2 = R_0\\
                \eta_2 C_1 + \rho_2 C_2 = J_0\\
            \end{array}
            \right \} \Rightarrow C1, C2
        }
    \end{equation*}
    Với $\lamba_1, \lamba_2$ tính từ (7), và $\rho_1, \rho_2$ được tính từ (9) hoặc (10) hoặc (11)  hoặc (12).

    \subsection{Problem 2}
    ...

    \subsection{Bonus exercises}
    ...

%%%%%%%%%%%%%%%%%%%%%%%%%%%%%%%%%


    \section{Probabilty}

    \subsection{Problem 1}
    ...

    \subsection{Problem 2}
    ...

    \subsection{Bonus exercises}
    ...

%%%%%%%%%%%%%%%%%%%%%%%%%%%%%%%%%


    \section{Graph}

    \subsection{Problem 1}
    ...

    \subsection{Problem 2}
    ...

    \subsection{Bonus exercises}
    ...

    \begin{thebibliography}{80}


        \bibitem{bib1}
        ...


        \bibitem{bib2}
        ...


    \end{thebibliography}
\end{document}

