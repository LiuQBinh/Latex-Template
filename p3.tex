%%%%%%%%%%%%%%%%%%%%%%%%%%%%%%%%%%%%%%
\newcommand{\hop}[1]{
    \begin{tcolorbox}
    [colback=cyan!15,colframe=orange!75!white]{#1}
    \end{tcolorbox}
}
\definecolor{lightgray}{gray}{.95}
%%%%%%%%%%%%%%%%%%%%%%%%%%%%%%%%%%%%%%%

\begin{document}

    \subsection{Bài tập 3:} Giả sử rằng tình yêu của Romeo và Juliet bị xáo trộn bởi các điều kiện bên ngoài. Trong trường hợp này, tình yêu giữa họ được mô hình hoá bởi Hệ phương trình vi phân như sau:


    \begin{equation}
        \systeme{
            R' = aR + bJ + f(t),
            J' = cR + dJ + g(t),
            R(0) = R_0,
            J(0) = J_0
        }\label{eq:equation2}
    \end{equation}

    Trong đó f và g là hai hàm thực phụ thuộc vào t. \\

    Để giải được hệ phương trình vi phân trên thì f(t) và g(t) phải có những dạng đặc biệt sau: \\

    $e^{\alpha t}.P_n(t)$ \\

    $e^{\alpha t}[P_n(t).sin\beta t +Q_m(t).cos\beta t]$ \\

    \textbf{Cách giải:}\\

    Từ 1 trong 2 phương trình rút 1 ẩn theo ẩn kia suy ra đẳng thức (*) \\

    Thế đẳng thức (*) vào phương trình còn lại suy ra phương trình vi phân cấp 2 hệ số hằng (**) \\

    Giải phương trình vi phân cấp 2 hệ số hằng (**) suy ra ẩn thứ nhất \\

    Thế ẩn thứ nhất vào đẳng thức (*) suy ra ẩn thứ hai.\\

    \textbf{Phương trình vi phân cấp 2:}\\
    1) Phương trình vi phân cấp 2 tuyến tính thuần nhất hệ số hằng:\\

    \begin{equation*}
        y'' + p.y' + q.y = 0
    \end{equation*}

    Xét phương trình đặc trưng:
    \begin{equation*}
        k^2 + p.k + q = 0
    \end{equation*}

    Nếu $\delta > 0$ \\
    $\Rightarrow$ phương trình có 2 nghiệm $k_1, k_2$ \\
    $\Rightarrow$ Phương trình (*) có nghiệm là: $y= C_1.e^{k_1 t}+C_2.e^{k_2 t}$ \\

    Nếu $\delta = 0$ \\
    $\Rightarrow$ Phương trình có nghiệm kép $k$ \\
    $\Rightarrow$ Phương trình (*) có nghiệm là: $y = C_1.e^{kt}+C_2.x.e^{kt}$ \\

    Nếu Nếu $\delta < 0$ \\
    $\Rightarrow$ Phương trình có nghiệm phức $\alpha \pm i.\beta$ \\
    $\Rightarrow$ Phương trình (*) có nghiệm là: $y = e^{\alpha t}.(C_1.cos{\beta t} + C_2.sin{\beta t})$ \\

    2) Phương trình vi phân cấp 2 tuyến tính hệ số hằng với vế phải đặc biệt: \\
    a)    Định lý: \\

    Giả sử: \\
    $ y''+p.y'+q.y=f(t)$
    có nghiệm $y=Y_1$ \\
    $y''+p.y'+q.y=g(t)$
    có nghiệm $y=Y_2$

    Khi đó: \\
    $y''+p.y'+q.y=f(t)+g(t)$
    có nghiệm $y=Y_1+Y_2$

    b) $y''+p.y'+q.y=e^{\alpha t}.P_n (t)$ với $P_n(t)$ là đa thức bận n\\

    Giải phương trình: $y''+p.y'+q.y=0 \Rightarrow y_c=Y$

    Nghiệm riêng $\alpha$ không là nghiệm của phương trình đặc trưng. \\
    $\Rightarrow y_p=e^{\alpha t}.Q_n (t)$ \\

    Nghiệm riêng $\alpha$ là nghiệm đơn của phương trình đặc trưng. \\
    $\Rightarrow y_p=t.e^{\alpha t}.Q_n (t)$ \\

    Nghiệm riêng $\alpha$ là nghiệm kép của phương trình đặc trưng. \\
    $\Rightarrow y_p=t^2.e^{\alpha t}.Q_n (t)$ \\

    Với $Q_n (t)$ là đa thức bậc n tổng quát \\
    $y''+p.y'+q.y=e^{\alpha t}.[P_n(t).sin{\beta t}+Q_m (t).cos{\beta t} ]$ \\

    Giải phương trình: \\
    $y''+p.y'+q.y=0 \Rightarrow y_c=Y$ \\

    Nghiệm riêng $\alpha \pm i \beta$ không là nghiệm của phương trình đặc trưng. \\
    $\Rightarrow y_p=e^{\alpha t}.[H_l (t).sin{\beta t}+K_l (t).cos{\beta t} ]$ \\

    Nghiệm riêng $\alpha \pm i \beta$ là nghiệm của phương trình đặc trưng. \\
    $\Rightarrow y_p=t.e^{\alpha t}.[H_l (t).sin{\beta t}t+K_l (t).cos{\beta t}βt ]$ \\

    Với $l=max⁡(n;m)$

    \textbf{Ví dụ 1}:

    \begin{equation}
        \systeme{
            R'=2R-J (1),
            J'=-2R+J+18t (2),
            R(0)=2,
            J(0)=-1
        }\label{eq:equation2}
    \end{equation}

    $(1) \Rightarrow J=-R'+2R (3) \\
    \Rightarrow J'=-R''+2R'$ \\

    Thế vào (2) ta được: \\
    $-R''+2R'=-2R-R'+2R+18t$ \\
    $\Rightarrow R''-3R'=-18t$        (4) \\

    Phương trình đặc trưng: \\
    $k^2 - 3k = 0 \\
    \Longleftrightarrow  k = 0 hoặc k = 3 \\
    \Rightarrow R_c = C_1 + C_2.e^{3t} \\
    $

    Ta có: \\
    $ -18t = e^{0t}.(-18t)$ \\
    $\Rightarrow \alpha = 0$ là nghiệm đơn của phương trình đặc trưng.\\
    Suy ra, nghiệm riêng của (4) có dạng: \\
    $ R_p=t.(At+B)=At^2+Bt \\
    \Rightarrow R_P'=2At+B \\
    \Rightarrow R_P''=2A \\
    $

    Thế vào (4) ta được:\\

    $2A-3.(2At+B)=-18t$ \\

    \begin{equation}
        \Longleftrightarrow
        \systeme{
            6At+2A-3B=-18t,
            2A - 3B = 0
        }\label{eq:equation2}
    \end{equation}

    \begin{equation}
        \Longleftrightarrow
        \systeme{
            A = 3,
            B = 3
        }\label{eq:equation2}
    \end{equation}
    $\Rightarrow R_p=3t^2+2t \\
    \Rightarrow R=C_1+C_2.e^3t+3t^2+2t \\
    \Rightarrow R'=3C_2.e^3t+6t+2 \\$

    Thế vào (3) ta được: \\

    $J=-(3C_2.e^{3t}+6t+2)+2(C_1+C_2.e^{3t}+3t^2+2t) \\
    \Rightarrow J=2C_1-C_2.e^{3t}+6t^2-2t-2
    $

    Ta có: \\

    \begin{equation}
        \systeme{
            R(0) = 2,
            J(0) = -1
        }\label{eq:equation2}
    \end{equation}

    \begin{equation}
        \Rightarrow
        \systeme{
            C_1 + C_2 = 2,
            2C_1 - C_2 -2 = -1
        }\label{eq:equation2}
    \end{equation}

    \begin{equation}
        \Longleftrightarrow
        \systeme{
            C_1 + C_2 = 2,
            2C_1 - C_2 =1
        }\label{eq:equation2}
    \end{equation}

    \begin{equation}
        \Longleftrightarrow
        \systeme{
            C_1 = 1,
            C_2 = 1
        }\label{eq:equation2}
    \end{equation}

    Vậy nghiệm của hệ phương trình:
    \begin{equation}
        \systeme{
            R = e^{3t}+3t^2+2t+1,
            J = -e^{3t}+6t^2-2t
        }\label{eq:equation2}
    \end{equation}

%%%%%%%%%%%%%%%%%%%%%%%%%%%%%%%%%%%%

    \textbf{Ví dụ 2}:

    \begin{equation}
        \systeme{
            R'=4R+6J    (1),
            J'=2R+3J+t   (2),
            R(0) = \frac{1}{7}\, J(0) = \frac{6}{49}
        }\label{eq:equation2}
    \end{equation}

    $
    (1) \Rightarrow 6J=R'-4R \\
    \Rightarrow J = \frac{1}{6} R^'- \frac{2}{3} R        (3) \\
    \Rightarrow J'= \frac{1}{6} R''- \frac{2}{3} R' \\
    $

    Thế vào (2) ta được:

    $
    \frac{1}{6}R'' - \frac{2}{3}R' = 2R + 3 (\frac{1}{6}R' - \frac{2}{3}R) + t \\
    \Longleftrightarrow R'' - 4R' = 12R + 3R' - 12R + 6t \\
    \Longleftrightarrow R'' - 7R' = 6t (4)
    $

    Phương trình đặc trưng:
    $
    k^2 - 7k = 0 \\
    \(\left[ \begin{array}{l}
                 k = 0\\k=7
    \end{array} \right.\\

    \Rightarrow R_C = C_1 + C_2.e^{7t} \\
    $

    Ta có:;
    $6t=e^{0t}.(6t) \\$
    $\Rightarrow$ $\alpha = 0$ là nghiệm đơn của phương trình đặc trưng.
    Suy ra, nghiệm riêng của (4) có dạng:
    $
    R_p=t.(At+B)=At^2+Bt \\
    \Rightarrow R_p'=2At+B \\
    \Rightarrow R_p''=2A \\
    $

    Thế vào (4) ta được:

    $
    2A-7.(2At+B)=6t
    \Longleftrightarrow -14At+2A-7B=6t
    $

    \begin{equation}
        \Longleftrightarrow
        \systeme{
            -14A = 6,
            2A - 7B = 0
        }\label{eq:equation2}
    \end{equation}

%%%%%%%%%%%%%%%%%%%%%%%%%%%%%%%5
    \textbf{Ví dụ 3:}

    \begin{equation}
        \systeme{
            R^'=4R-3J+sin⁡t         (1),
            J^'=2R-J-2 cos⁡t    (2),
            R(0)=0,
            J(0)=2
        }\label{eq:equation2}
    \end{equation}

    $
    (2) \Rightarrow 2R=J'+J+2 cost \\
    \Rightarrow R= \frac{1}{2} J'+ \frac{1}{2}J+cos⁡t         (3) \\
    \Rightarrow R'= \frac{1}{2} J''+\frac{1}{2} J'-sin⁡t\\
    $

    Thế vào (1) ta được: \\

    $
    \frac{1}{2} J''+ \frac{1}{2} J'- sint = 4.(\frac{1}{2}J'+\frac{1}{2}J+ cost) - 3J + sint \\
    \Longleftrightarrow J'' + J' - 2sint = 4J'+4J+8 cos⁡t-6J+2 sin⁡t\\
    \Longleftrightarrow J''-3J'+2J=4 sin⁡t+8 cos⁡t         (4) \\
    $

    Phương trình đặc trưng: \\
    $
    k^2-3k+2=0
    \Longleftrightarrow \(\left[ \begin{array}{l}
                                     k=1\\k=2
    \end{array} \right.\ \\
    \Rightarrow J_C = C_1.e^t + C_2.e^2t \\
    $

    Ta có: \\
    $4sint+8cost=e^{0t}.(4sint+8cost)$ \\
    $\Rightarrow \alpha \pm i\beta = \pm i$ không là nghiệm của phương trình đặc trưng. \\
    Suy ra, nghiệm riêng của (4) có dạng: \\
    $
    J_P=A.sin⁡t+B.cos⁡t \\
    \Rightarrow J_P'=A cos⁡t-B sin⁡t\\
    \Rightarrow J_P''=-A sin⁡t-B cos⁡t \\
    $

    Thế vào (4) ta được: \\

    $
    (-A+3B+2A)  sin⁡t+(-B-3A+2B)  cos⁡t=4 sin⁡t+8 cos⁡t \\
    \Longleftrightarrow (A+3B)  sin⁡t+(-3A+B)  cos⁡t=4 sin⁡t+8 cos⁡t\\
    \Longleftrightarrow
    $
    \begin{equation}
        \systeme{
            A + 3B = 4,
            -3A + B = 8
        }\label{eq:equation2}
    \end{equation}
    \begin{equation}
        \Longleftrightarrow
        \systeme{
            A = -2,
            B = 2
        }\label{eq:equation2}
    \end{equation}
\end{document}
